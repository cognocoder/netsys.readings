
\chapter{P4: Programming Protocol-Independent Packet Processors}
\fullcite{bosshart2014p4} \cite{bosshart2014p4}.


\phantomsection
\section*{Introdução}
\addcontentsline{toc}{section}{Introdução}

Redes Definidas por Software possuem plano de controle fisicamente separado do plano de dados de um conjunto de \textit{switchs} em uma mesma rede. O modelo OpenFlow estabeleceu uma interface de aplicação programável única e independente de hardware, mas sua especificação cresceu em complexidade na medida em que novos campos de cabeçalho foram inseridos ao longo dos anos. O artigo propõe uma linguagem de alto nível para a programação independente de protocolo para processamento de pacotes para permitir flexibilidade no plano de controle, de modo que possa determinar como que \textit{switchs} devem processar pacotes independente de hardware e em ambiente de produção.


\phantomsection
\section*{Modelo Abstrato de Encaminhamento}
\addcontentsline{toc}{section}{Modelo Abstrato de Encaminhamento}

Diferente do modelo OpenFlow, este artigo assume processamento flexível de pacotes, com estágios de correspondência e ação em paralelo e em série, onde ações são compostas por primitivas independente de protocolo definidas pelo \textit{switch}. O programa de processamento de pacotes definido pelo operador de rede é compilado para a plataforma de destino, processo no qual é responsável por utilizar recursos disponíveis em hardware.

O modelo abstrato de encaminhamento possui dois tipos de operações: configuração e povoamento. A operação de configuração consiste em determinar quais protocolos são suportados e como um \textit{switch} processa os pacotes, através dos estágios de correspondência e ações. A operação de povoamento, por sua vez, consiste na inserção ou remoção de entradas nas tabelas de encaminhamento. O artigo assume que tais operações são realizadas em fases distintas, mas defende que implementações devem ser capazes de alterar plano de encaminhamento sem interrupção do serviço.

Uma vez processado, um pacote é direcionado às tabelas de correspondência e ações, divididas em pares de entrada e saída, com suporte a adição de metadados.


\phantomsection
\section*{Programação e Compilação}
\addcontentsline{toc}{section}{Programação e Compilação}

A linguagem proposta pelo artigo permite a definição de tipos legais em cabeçalho de pacotes, descrião de regras de correspondência e ação para tabelas de encaminhamento dinâmicas, e execução de estágios em paralelo e em série.

O artigo propõe um processo de compilação em duas etapas. Na primeira estapa, programadores estabelecem o fluxo de controle do programa através de uma liguagem imperativa. Em seguida, o compilador traduz tal fluxo de controle em uma estrutura em grafo, de modo a facilitar a análise de dependência entre estágios executados em série e estágios paralelos.

Os principais componentes da linguagem proposta são definições de: cabeçalhos, com sequência e comprimento de campos; processos de análise de cabeçalhos; tabelas de correspondência e ações; ações compostas por primitivas simples; e programas de controle, que estabelecem fluxo de controle de maneira imperativa entre tais tabelas.

O compilador é responsável por analisar a descrião das tabelas de correspondência e ações e alocar os recursos necessários em hardware. Caso a plataforma de destino possua capacidade de processamento programável, ele traduz o processo de análise em uma máquina de estados.


\phantomsection
\section*{Conclusão}
\addcontentsline{toc}{section}{Conclusão}

Este trabalho, no entanto, não aborda diversos aspectos operacionais de um \textit{switch} --- primitivas de controle de congestionamento, técnicas de enfileiramento, entre outros ---, mas consiste em um passo em direção à flexibilização do plano de controle em Redes Definidas por Software, através de linguagem imperativa de alto nível programação independente de protocolo para processamento de pacotes. Operadores definem como \textit{switchs} processam pacotes através desta linguagem que então é compilada e implementado em harware específico.
