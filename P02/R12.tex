
\chapter{RSS++ load and state-aware receive side scaling}
\fullcite{barbette2019rsspp} \cite{barbette2019rsspp}.


\phantomsection
\section*{Introdução}
\addcontentsline{toc}{section}{Introdução}

Aplicações de rede frequentemente usam múltiplos núcleos de CPU para lidar com 
alta demanda e apresentar baixa latência. O estado da arte, conforme consta no 
trabalho, relata o uso da fragmentação de recursos com agregação por fluxo. Em 
tal técnica, um pacote resulta em carga de trabalho um em núcleo de CPU, 
determinado por fluxo, de modo que a distribuição não balanceada de carga pode 
resultar em enfileiramento de pacotes, aumento da latência, e, em última 
instância, no descarte de pacotes. O superdimensionamento de recursos 
computacionais não é adequado para endereçar uma solução neste contexto.

A fragmentação se trata de uma tarefa não trivial, com os seguintes requisitos: 
distribuição de carga, afinidade de fluxo por fragmento, rearranjo mínimo de 
fluxos e seus estados entre fragmentos.


\phantomsection
\section*{Projeto}
\addcontentsline{toc}{section}{Projeto}

A ferramenta propõe uma solução para o problema de pacotes através do cálculo 
de função \textit{hash} sobre seu cabeçalho, cujo resultado determina em qual 
fragmento o fluxo é alocado. A carga é balanceada por um algoritmo de 
otimização, por uma tabela compartilhada entre os núcleos que relaciona fluxos 
e fragmentos. \texttt{RSS++} permite a alocação dinâmica de núcleos e previne a 
reordenação de pacotes durante a migração de fragmentos entre núcleos.

A solução é aplicável quando há entropia entre os campos do cabeçalho e fluxos 
demandam aproximadamente a mesma carga de recursos computacionais.


\phantomsection
\section*{Implementação e Avaliação}
\addcontentsline{toc}{section}{Implementação e Avaliação}

Apresentou-se duas implementações: em \texttt{DPDK} e no kernel do Linux, 
através de programa \texttt{BPF} inserido no \texttt{XDP}. Para avaliar a 
ferramenta, um estudo de caso foi realizado sobre base de tráfego coletado em 
um campus. O objetivo foi responder às seguintes perguntas de pesquisa:

A avaliação constatou distribuição justa o suficiente da carga, com redução na 
latência em uma ordem de grandeza. Falta do estado de fluxo pode demandar 
reordenação de pacotes, bem como o uso de tabelas por fragmentos resultou em 
maior taxa de vazão e menor latência. A ferramenta melhora o uso de recursos 
computacionais e apresenta melhor desempenho.


\phantomsection
\section*{Conclusão}
\addcontentsline{toc}{section}{Conclusão}

O trabalho lista limitações e retrições em cenários ainda não explorados, mas 
que consistem em trabalhos futuros, como: o processamento e análise de 
desempenho sobre séries de pacotes pequenos e séries de pacotes grandes, 
inter-conexão de memórias entre núcleos do dispositivo de rede, suporte a 
múltiplas aplicações, implementação em clientes. Este trabalho permite a 
distribuição de carga escalável em mútiplos núcleos em um servidor.
