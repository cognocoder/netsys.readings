
\chapter{TCP slow start, congestion avoidance, fast retransmit, and fast recovery algorithms}
\fullcite{stevens1997tcp} \cite{stevens1997tcp}.


\phantomsection
\section*{Slow-Start}
\addcontentsline{toc}{section}{Slow-Start}

Embora a implementação do fluxo de transmissão segundo anúncio do destinatário 
funcione em situações restritas, sua aplicabilidade a redes nas quais fluxo é 
variante e há elementos com links lentos resultam em congestionamentos.

O algoritmo \textit{slow-start} apresenta solução para este problema. A janela 
de transmissão é inicialmente de 1 segmento. Sempre que um segmento de 
confirmação é recebido ele é incorporado à janela, que é expandida 
exponencialmente. A perda de pacotes sinaliza que a janela está larga.


\phantomsection
\section*{Prevenção de Congestionamento}
\addcontentsline{toc}{section}{Prevenção de Congestionamento}

Congestionamentos ocorrem quando há maior fluxo de entrada do que de saída. A 
prevenção de congestionamento lida com a perda de pacotes.

O tamanho da janela de transmissão é o mínimo entre o anunciado pelo 
destinatário e o determinado pelo \textit{slow-start}. Em caso de 
congestionamento, a janela é reduzida pela metade; a 1 segmento, em caso de 
\textit{timeout}. Este novo valor é salvo em uma variável de limiar, com tamanho 
mínimo de 2 segmentos.

Se a janela for menor ou igual ao limiar, a janela é expandida com 
\textit{slow-start}. Caso contrário, seu crescimento é linear.


\phantomsection
\section*{Retransimssão Rápida}
\addcontentsline{toc}{section}{Retransimssão Rápida}

Duplicadas do pacote de confirmação podem ocorrer caso os pacotes sejam 
recebidos fora de ordem. Na detecção de três ou mais duplicadas, é entendido a
perda de um pacote e o TCP realiza sua retransmissão.


\phantomsection
\section*{Recuperação Rápida}
\addcontentsline{toc}{section}{Recuperação Rápida}

No caso de retransmissão rápida, a prevenção de congestionamento é executada. 
Isto é a recuperação rápida.
