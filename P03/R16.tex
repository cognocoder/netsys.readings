
\chapter{Modeling TCP throughput: A simple model and its empirical validation}
\fullcite{padhye1998modeling} \cite{padhye1998modeling}.


\phantomsection
\section*{Introdução}
\addcontentsline{toc}{section}{Introdução}

O trabalho propõe caracterização analítica do TCP em função da taxa de perda e 
do RTT para transferência em massa. Avaliou-se o impacto de \textit{timeouts}, 
que ocorrem com maior frequência do que a retransmissão rápida, bem como do 
tamanho da janela do receptor.

Um modelo analítico do TCP oferece possibilidade da divisão justa de banda e a 
interação entre fluxo de outros protocolos e a conexão TCP.


\phantomsection
\section*{Modelo, Experimentos e Análise}
\addcontentsline{toc}{section}{Modelo, Experimentos e Análise}

O trabalho apresenta um modelo estocástico para a prevenção de congestionamento 
TCP com uma expressão relativamente simples. Considera como indicador de perda 
de pacotes o recebimento de três ou mais duplicatas de pacotes de confirmação 
ou \textit{timeout}. O modelo também incorpora o tamanho de janela do receptor.

A avaliação se deu através da experimentação e análise com bases de dados, nas 
quais constatou-se a validade do modelo, bem como a predominância de 
\textit{timeouts} em relação às duplicatas de pacotes de confirmação.


\phantomsection
\section*{Discussão e Conclusão}
\addcontentsline{toc}{section}{Discussão e Conclusão}

O modelo é compatível com comportamento observado na maior parte dos casos. Em 
especial, destaca-se o impacto dos \textit{timeouts} na taxa de transmissã. A 
expressão permite a interação entre fluxo de outros protocolos e a conexão TCP.

Os autoreas apresentam possibilidades de trabalhos futuros e discutem limites e 
premissas: o trabalho não captura sutilezas da recuperação rápida, por exemplo.
