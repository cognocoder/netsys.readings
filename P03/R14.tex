

\chapter{Analysis of the increase and decrease algorithms for congestion avoidance in computer networks}
\fullcite{chiu1989analysis} \cite{chiu1989analysis}.


\phantomsection
\section*{Introdução}
\addcontentsline{toc}{section}{Introdução}

O fenômeno de congestionamento de redes se tornou frequente a partir da 
interconexão por dispositivos heterogeneos, com especificações de desempenho 
distintas. Caso um dispositivo recebe mais mensagens do que pode processar, 
pacotes pendentes são armazenados em filas. A partir do momento em que o 
tamanho dessas filas é ultrapassado, pacotes são descartados, fato que só é 
perceptível no rementente uma vez que o tempo para o recebimento do pacote de 
confirmação expirar. O aumento do tempo de resposta é utilizado como 
diagnóstico para o congestionamento.

O controle e prevenção de congestionamento tratam da gerência do uso de 
recursos e operam no limiar delineado acima ao ajustar a janela ou taxa de 
transmissão. A análise dos algoritmos se dá em uma perspectiva além da 
prevenção de congestionamento. Os controladores são avaliados em termos da 
eficiência, equidade, distributividade e convergência.


\phantomsection
\section*{Controles Lineares}
\addcontentsline{toc}{section}{Controles Lineares}

A análise do sistema linear é realizada por mapa de demanda de agentes por 
recursos da rede, com convergência diagramada vetorialmente. A sua eficiência de 
convergência é analisada de acordo com sua responsividade e suavidade --- 
antagônicos entre si --- em relação ao ponto objetivo.


\phantomsection
\section*{Controles Não-Lineares}
\addcontentsline{toc}{section}{Controles Não-Lineares}

Embora controles não-lineares ofereçam maior flexibilidade na busca por 
equidade da distribuição de recursos da rede, precisar os fatores do modelo é 
complexo. Ademais, a sensibilidade do controle não-linear a esses parâmetros o 
caracteriza com robustez reduzida.


\phantomsection
\section*{Conclusões}
\addcontentsline{toc}{section}{Conclusões}

A escolha de um algoritmo independente de escala ou parâmetros de software e 
hardware se faz necessária pela heterogeneidade dos dispositivos de rede, bem 
como a operação sobre fatores integrais. A facilidade de implementação é
desejável. O trabalho propõe as seguintes perguntas:
\begin{enumerate}
  \item Como o atraso de retorno afeta o controle?
  \item Qual a utilidade de aumentar o tamanho do campo de retorno?
  \item É vantajoso estimar o número de clientes?
  \item Qual o impacto da operação assíncrona?
\end{enumerate}

O trabalho analisa conjunto de algoritmos de controle e prevenção de 
congestionamento que visa minimizar a latência e maximizar a janela ou taxa de 
transmissão através de retorno do estado da rede a usuários, que ajustam 
demanda por recursos da rede. Estabelece condições pelas quais se pode manter 
ou melhorar métricas. A melhor política de controle e prevenção de 
congestionamento é o aumento aditivo e a redução multiplicativa.
