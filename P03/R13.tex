
\chapter{Congestion avoidance and control}
\fullcite{jacobson1988congestion} \cite{jacobson1988congestion}.


\phantomsection
\section*{Introdução}
\addcontentsline{toc}{section}{Introdução}

O crescimento das redes de computadores resultaram en congestionamento de 
tráfego. Foram propostos algoritmos em contraponto à conservação de pacotes, 
que se mostrou não robusta, na medida em que se observa que a conexão pode não 
atingir um ponto de equilíbrio, um remetente pode inserir um novo pacote na 
rede antes que um antigo saia --- de forma a extrapolar a janela de transmissão 
---, ou o equilíbrio não é possível de ser atingido em função de limitações no
caminho.


\phantomsection
\section*{Equilíbrio com \textit{Slow-Start}}
\addcontentsline{toc}{section}{Equilíbrio com \textit{Slow-Start}}

No contexto de fluxos conservativos, nos quais a densidade de pacotes entre 
remetente e destinatário é constante, falhas ocorrem de conexões que começam ou 
recomeçar após uma perda de pacote. Após o envio de uma janela de pacotes, o 
remetente está sujeito ao tempo de envio no link mais lento e ao recebimento de 
pacotes de confirmação. Como o destinatário só pode enviar um pacote de 
confirmação após receber um pacote do remetente, este protocolo de transporte é 
dito auto-sincronizado. Para determinar o fluxo de dados é necessário 
determinar a sincronização, e vice-versa.

O algoritmo \textit{slow-start} consiste em determinar o tamanho da janela de 
trasmissão --- inicialmente de 1 pacote --- e incrementar o tamanho na medida 
em que pacotes de confirmação são recebidos: se a rede comporta o pacote de 
confirmação, a janela pode ser incrementada. O tamanho da janela utilizado é o 
valor mínimo entre a janela determinada pelo \textit{slow-start} e o anunciado 
pelo destinatário. A progressão da janela é exponencial e seu efeito na 
performance é negligenciável.


\phantomsection
\section*{Conservação no Equilíbrio}
\addcontentsline{toc}{section}{Conservação no Equilíbrio}

Após implementar o fluxo de dados, se assumirmos que o protocolo foi 
implementado corretamente, se faz necessário estimar o RTT médio e sua 
variância. O trabalho propõe método para estimar a variância com base em uma 
relação com entre sinal e ruído. A média é calculada a partir de relação 
especificada pelo padrão TCP, ambas as estimativas amparadas pela teoria de 
controle.

Em caso de falhas, o espaçamento para retransmissão é exponencial. Isso ocorre 
pelo fato de que redes se comportam como sistemas lineares: atingem 
estabilidade em tempo exponencial, se estáveis; podem se tornar estáveis se 
aplicarmos atenuação exponencial de seus estímulos de entrada --- espaçamento 
exponencial ---, se instáveis.


\phantomsection
\section*{Controle de Congestionamento}
\addcontentsline{toc}{section}{Controle de Congestionamento}

Uma vez estabelecidos o fluxo de transmissão e o equilíbrio em função da 
variação no tempo de envio, se faz necessário estabelecer uma estratégia de 
controle de congestionamento.
Como a perda de pacotes por dano é relativamente rara --- para conexões 
cabeadas, à época do artigo ---, a perda de pacotes é utilizada como sinal de 
que a rede se encontra congestionada.

O remetente deve expandir a janela de transmissão caso não receba sinal de que 
a rede está congestionada. Processo que se dá como no \textit{slow-start} e 
usualmente é implementado de forma conjunta, mas são dois processos distintos. 
A expansão se dá através da soma de pacote de confirmação --- incremento de uma 
unidade.

O remetente deve retrair a janela de trasmissão caso receba sinal de que a rede 
está congestionada. Processo que se dá pela multiplicação de um fator $k$, 
$0 \le k \le 1$ --- 0,5 praticado. Os remetentes precisam aliviar a demanda de 
recursos da rede para descaracterizar a rede como descongestionada.


\phantomsection
\section*{Trabalhos Futuros}
\addcontentsline{toc}{section}{Trabalhos Futuros}

Os algoritmos apresentados garatem que a capacidade do link não será excedida, 
mas não garante compartilhamento justo. Somente nos pontos aonde ocorrem 
convergência de fluxo é possível prover solução para este problema. A deteção 
de congestionamentos em estágio prematuro se faz importante, dado seu 
agravamento exponencial, passível de solução através do ajuste da janela de 
transmissão. Este problema se mostra não-trivial, dado a natureza explosiva do 
tráfego de dados; no entanto.
